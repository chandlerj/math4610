\documentclass[10pt, AMS Euler]{article}
\textheight=9.25in \textwidth=7in \topmargin=-.75in
 \oddsidemargin=-0.25in
\evensidemargin=-0.25in
\usepackage{url}  % The bib file uses this
\usepackage{graphicx} %to import pictures
\usepackage{amsmath, amssymb}
\usepackage{theorem, concrete, multicol, color}


\setlength{\intextsep}{5mm} \setlength{\textfloatsep}{5mm}
\setlength{\floatsep}{5mm}
\setlength{\parindent}{0em} % new paragraphs are not indented


%%%%  SHORTCUT COMMANDS  %%%%
\newcommand{\ds}{\displaystyle}
\newcommand{\Z}{\mathbb{Z}}
\newcommand{\arc}{\rightarrow}
\newcommand{\R}{\mathbb{R}}
\newcommand{\N}{\mathbb{N}}
\newcommand{\Q}{\mathbb{Q}}
\newcommand{\blank}{\underline{\hspace{0.33in}}}
\newcommand{\qand}{\quad and \quad}
\newcommand{\stirling}[2]{\genfrac{\{}{\}}{0pt}{}{#1}{#2}}
\newcommand{\dydx}{\ds \frac{d y}{d x}}
\newcommand{\ddx}{\ds \frac{d}{d x}}
\newcommand{\dvdx}{\ds \frac{d v}{d x}} 

%%%%  footnote style %%%%

\renewcommand{\thefootnote}{\fnsymbol{footnote}}

\pagestyle{empty}

\begin{document}

\begin{flushright}
Chandler Justice - A02313187
\end{flushright}
\noindent \underline{\hspace{3in}}\\

\textit{Iterative methods for solutions of linear systems of equations}\\

Lets say we have an equation of the form $Ax =b$, we can decompose the matrix $A$ into the following form:

\begin{center}
  $
  L =
  \begin{bmatrix}
    0 & 0 & ... & 0\\
    a_{21} & 0 & ... & 0\\
    a_{31} & a_{32} & ... & 0\\
    ... & ... & ... & ...\\
    a_{n-1} & ... & a_{n, n-1} & 0\\
  \end{bmatrix}
  D = 
  \begin{bmatrix}
    a_{11} & 0 & ... & 0\\
    0 & a_{21} & ... & 0\\
    ... & ... & ... & ...\\
    ... & ... & ... & a_{nm}\\
  \end{bmatrix}
  $\\
  ...and an upper triangular matrix; that I am not going to write but you get the idea.\\
\end{center}

\textit{Theorem:} If $A$ is a diagonally dominant matrix then jacobi iteration converges to the solution of $Ax = b$.\\

$
x_1 = D^{-1}(b - (L + U)x_0)\\
x_{k+1} = D^{-1}(b - (L + U)x_k)\\
x_{k+m} = D^{-1}(b - (L + U)x_k) = D^{-1}(b - (L + D + U)x_k + Dx_k)\\
\rightarrow = D^{-1}(b - Ax_k + Dx_k) = D^{-1} r_k + x_k = x_k + D^{-1}r_k\\
$\\


\textbf{Note:} $b - Ax_k$ is he residual vector. Recall the residual is defined as:\\
$Ax = b \rightarrow r = b - Ax$\\

We can use this to create the conditional: if $||r_k||_2 \leq tol \rightarrow$ STOP.\\

\textit{Jacobi Iteration}\\

\textbf{Inputs:} $A, x_0$\\

\textbf{Loop:}\\
$r = b - Ax \quad$ for all $k = 0,1,2,..$\\
$x_{k+1} = x_k + D^{-1} r_k$\\
error = $|| b - Ax_k||$\\
$x_k = x_{k + 1}$\\

Since $D$ is a diagonal matrix, $D^{-1}$ will just be $\frac{1}{D}$, which gives us our $x_{k+1}$ modified matrix.\\

Lets talk more about some tricks with the residual:\\
$
r_{k+1} = b - Ax_{k+1} = b - A(x_k + D^{-1}r_k)\\
\rightarrow = (b - Ax_k) - AD^{-1}r_k\\
$

\textit{Guass-Seidel}\\

$
A = (L + D + U)\\
Ax = b\\
(D + U)x = b - Lx\\
(D + U)x_{k+1} = (b - Lx)\rightarrow (D + U)x_{k + 1} = (b - Lx_k)\\
x = (D + U)^{-1} (b - Lx) \\
x^{(k + 1)} = (D + U)^{-1} (b - Lx^k)\\
$\\

We can use our Back-substitution routine to find $(D + U)^{-1}$\\



\noindent \underline{\hspace{3in}}\\

\end{document}

