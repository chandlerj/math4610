\documentclass[10pt, AMS Euler]{article}
\textheight=9.25in \textwidth=7in \topmargin=-.75in
 \oddsidemargin=-0.25in
\evensidemargin=-0.25in
\usepackage{url}  % The bib file uses this
\usepackage{graphicx} %to import pictures
\usepackage{amsmath, amssymb}
\usepackage{theorem, concrete, multicol, color}


\setlength{\intextsep}{5mm} \setlength{\textfloatsep}{5mm}
\setlength{\floatsep}{5mm}
\setlength{\parindent}{0em} % new paragraphs are not indented


%%%%  SHORTCUT COMMANDS  %%%%
\newcommand{\ds}{\displaystyle}
\newcommand{\Z}{\mathbb{Z}}
\newcommand{\arc}{\rightarrow}
\newcommand{\R}{\mathbb{R}}
\newcommand{\N}{\mathbb{N}}
\newcommand{\Q}{\mathbb{Q}}
\newcommand{\blank}{\underline{\hspace{0.33in}}}
\newcommand{\qand}{\quad and \quad}
\newcommand{\stirling}[2]{\genfrac{\{}{\}}{0pt}{}{#1}{#2}}
\newcommand{\dydx}{\ds \frac{d y}{d x}}
\newcommand{\ddx}{\ds \frac{d}{d x}}
\newcommand{\dvdx}{\ds \frac{d v}{d x}} 

%%%%  footnote style %%%%

\renewcommand{\thefootnote}{\fnsymbol{footnote}}

\pagestyle{empty}

\begin{document}

\begin{flushright}
Chandler Justice - A02313187
\end{flushright}
\noindent \underline{\hspace{3in}}\\
\textit{November 27, 2023}\\
\textit{Where we are going with the class}
\begin{itemize}
  \item Complete Assn. 7 today
  \item Assn 8 - Using Jacobi to compute iteration solns to $Ax =b$
  \item Assn 9 - Speed code up using OpenMP
\end{itemize}
\textit{Iterative methods to computing solutions: Jacobi iteration \& more}\\
let: $A = L + D + U$\\
$Ax = b \rightarrow (L + D + U) x = b \rightarrow Dx = b - (L + U)x$\\
$x = D^{-1}(b- (L + U) x)$\\
\begin{itemize}
  \item $L$: A lower triangular matrix with zeros on the diagonal
  \item $D$: A Matrix with non-zero values only in the diagonal
  \item $U$: An upper triangular matrix with zeros on the diagonal
\end{itemize}
$x^{k+1} = D^{-1} (b - (L + U)x^k)$\\
$x^{k+1} \rightarrow Ax^* = b$\\
$x^{k+1} \rightarrow x^*$\\
\textbf{Note:} $A$ is diagonally dominant\\

\textbf{Implementation for Jacobi}\\
$Ax + b \rightarrow x^{k+1} = D^{-1}(b - (L + U)x^k) \rightarrow x^*$\\
\textit{input:} $A \in \R^{n x m}, b \in \R^n, x^0 \in \R^n$, tol maxIter\\
\textit{initialize:} error = 10 * tol; iter = 0;\\

\textit{loop:}\\
while (error $>$ tol \&\& iter $<$ maxIter)\{\\
for(int i = 0; i < n; i++)\{\\
double sum = $b_i$;\\
for(int j = 0; j < n; j++)\{\\
sum += $a_{i,j} * x_0$;\\
for(int j = i + 1; j < n; j++)\{\\
sum += $a_{i,j} * x_0$\}\\
$x_i = sum / a_{i,i}$;\}\\
double error = 0.0;\\
for(int i = 0; i < n; i++)\{\\
double value = $x1_i - x0_i$;\\
$error += val + val_i$\}\\
error = sqrt(error);\\
iter++;\}\\

\textit{November 29, 2023}\\

\textbf{Guass-Seidel Iteration}\\
$A = (L + D + U) \rightarrow Ax = b$\\
$(D + U) x = b - L x$\\
$x = (D + U)^{-1} (b - L x)$\\
$x^{(k+1)} = (D + U)^{-1} (b - Lx^k)$\\
$(D + U)^{-1} x$ (Use back substitution to obtain $(D + U)^{-1}$)\\
Residual formula: $x^{k+1} = (D + U)^{-1} (b - (L + D + U - D - U) x^k)$\\
$\rightarrow (D + U)^{-1} (b - Ax^k + (D + U)x^k) \rightarrow x^k + (D + U)^{-1} r^k$\\
$r^k = b - Ax^k$\\

\textit{Example - Leslie Matrix}\\
$\L =
\begin{bmatrix}
  f_0 & f_1 & ... & f_n\\
  s_0 & 0 & ... & 0\\
  0 & s_1 & ... & 0\\
  ...\\
  0 & ... & s_{n-1} & 0\\
\end{bmatrix}\\
$\\
$LN_0 \rightarrow N_1 \rightarrow N_0 = L^{-1} N $\\
We can decompose this matrix into three separate matrices $L, U, D$. We cannot apply iterative methods to finding solutions for this decomposition (or the original matrix) because there are too many zero values, which have an inverse of $\infty$.\\

\textit{Time complexity of iterative solvers}
\begin{itemize}
  \item GE + BS $\approx O(n^3) + O(n^2) \rightarrow O(n^3)$
  \item LU + FS + BS $\approx O(n^3) + O(n^2) + O(n^2)$
  \item Jacobi $\approx O(n^2) /$ iteration
  \item Guass-Seidel $\approx O(n^2) /$ iteration
\end{itemize}

\textit{December 1, 2023}\\

\textit{Guass-seidel elimination cont.}\\

$x^{k+1} = x^k + "(D + U)^{-1}" r^k$\\
$x^0, x^1, x^2, ... \rightarrow Ax^k = b$\\

\textit{LU-factorization}\\
Let $A \in \R^{n x m}$\\
\textbf{note:} $A$ is diagonally dominant\\


\textbf{some table koebbe wrote on the board}\\
\begin{tabular}{c c c}
    LU-factorization & jacobi & guass-seidel\\
    num. flops and error & num. flops and error & \textbf{ditto}\\
    
\end{tabular}\\

We need to compare and constrast the results of these metrics to determine which approach is best given the input.\\

\textit{Overview of upcoming homework}\\
\begin{itemize}
    \item hw08: due at the end of finals week
    \item Using OpenMP to parallelize $2$ linear solving algorithms (will use code from hw08). Due at end of finals week
\end{itemize}
\noindent \underline{\hspace{3in}}\\

\end{document}

