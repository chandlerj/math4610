\documentclass[10pt, AMS Euler]{article}
\textheight=9.25in \textwidth=7in \topmargin=-.75in
 \oddsidemargin=-0.25in
\evensidemargin=-0.25in
\usepackage{url}  % The bib file uses this
\usepackage{graphicx} %to import pictures
\usepackage{amsmath, amssymb}
\usepackage{theorem, concrete, multicol, color}


\setlength{\intextsep}{5mm} \setlength{\textfloatsep}{5mm}
\setlength{\floatsep}{5mm}
\setlength{\parindent}{0em} % new paragraphs are not indented


%%%%  SHORTCUT COMMANDS  %%%%
\newcommand{\ds}{\displaystyle}
\newcommand{\Z}{\mathbb{Z}}
\newcommand{\arc}{\rightarrow}
\newcommand{\R}{\mathbb{R}}
\newcommand{\N}{\mathbb{N}}
\newcommand{\Q}{\mathbb{Q}}
\newcommand{\blank}{\underline{\hspace{0.33in}}}
\newcommand{\qand}{\quad and \quad}
\newcommand{\stirling}[2]{\genfrac{\{}{\}}{0pt}{}{#1}{#2}}
\newcommand{\dydx}{\ds \frac{d y}{d x}}
\newcommand{\ddx}{\ds \frac{d}{d x}}
\newcommand{\dvdx}{\ds \frac{d v}{d x}} 

%%%%  footnote style %%%%

\renewcommand{\thefootnote}{\fnsymbol{footnote}}

\pagestyle{empty}

\begin{document}

\begin{flushright}
Chandler Justice - A02313187
\end{flushright}
\noindent \underline{\hspace{3in}}\\

\textit{December 6, 2023}\\

\textbf{A closer look at OpenMP}\\

\begin{itemize}
    \item all preprocessor directives must include \#pragma omp parallel num\_threads(n)
    \item surround parallel regions in curly braces so OMP does not stop parallelizing prematurely
    \item can use "\#pragma\ omp\ for" to automatically parallelize a for loop already inside of a parallel region
    \item If the item you want to parallelize is directly below the directive, creating a new scope is not unnecessary
    \item OMP uses a fork/join model to distribute work, wherein all threads branch (fork) off from a master and then each thread computes its own work, and then the partial work computed by the sums is joined back together into the master thread
    \item \textit{intrinsic functions:} use printf("\%d", omp\_get\_thread\_num()) to verify the number of threads you think are running are actually running.
    \item use omp\_get\_wtime() to measure the wall clock time. This will account for the fact that multiple threads are running at the same time. Using a traditionl approach to timing the wall time would result in a different time being computed by every thread.
\end{itemize}

\noindent \underline{\hspace{3in}}\\

\end{document}

